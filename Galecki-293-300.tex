\section{Linear Mixed Effects Models}
\subsection{Extracting Information from a Model-Fit Object of class lme.}
% Page 293-295 Galecki

%-------------------------------------------------------------%
% Page 293 

Galecki discusses several method to extract from a model fit of class lme.
For an R object lme.ft, which contains the results of fitting a single level LME.
By default, level=0.95 is ised.
%-------------------------------------------------------------%
% Page 294 
\begin{itemize}
\item ML Value logLik(lme.fit,REML=FALSE)
\item REML Value logLik(lme.fit,REML=FALSE)
\item AIC(lme.fit) AIC(lme.fit)
\item BIC(lme.fit) BIC(lme.fit)
\end{itemize}

%-------------------------------------------------------------%
%Page 295 
\begin{description}
\item[interval()]: computes confidence intervals for the model parameters, with named components.
\item[fixef()]: fixed effects
\item[ranef()]: by applying the function ranef() to a lme class model-fit object, the estimated random
effects are displayed. By default, the effects of all levels of the groupings are displayed.
\item[coef()]:  (help: ?coef.lme)
\item[corStruct()]: 
\item[getVarCov()]:
\item[VarCorr()]
\end{description}

%-------------------------------------------------------------%
%PAge 296
predict() : new data


\subsection{Tests of Hypothesis about Model Parameters}
MArginal approach tests, the argument type marginal should be used.



anova() is applied to two or more objects of the class lme it provides LR statistics, calculated on consecutive pairs of objects.



%Page 299 The function simulate.


%------------------------------------------------------------------
exactRLRM
\end{document}
