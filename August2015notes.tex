In statistics, deviance is a quality of fit statistic for a model that is often used for statistical hypothesis testing. It is a generalization of the idea of using the sum of squares of residuals in ordinary least squares to cases where model-fitting is achieved by maximum likelihood.

%-----------------------------------%
Create a basic lme4 Data Set
 
method 


%====================================================================%
In statistics, the predicted residual sum of squares (PRESS) statistic is a form of cross-validation used in regression analysis to provide a summary measure of the fit of a model to a sample of observations that were not themselves used to estimate the model. It is calculated as the sums of squares of the prediction residuals for those observations.[1][2][3]
A fitted model having been produced, each observation in turn is removed and the model is refitted using the remaining observations. The out-of-sample predicted value is calculated for the omitted observation in each case, and the PRESS statistic is calculated as the sum of the squares of all the resulting prediction errors:[4]
\operatorname{PRESS} =\sum_{i=1}^n (y_i - \hat{y}_{i, -i})^2 
Given this procedure, the PRESS statistic can be calculated for a number of candidate model structures for the same dataset, with the lowest values of PRESS indicating the best structures. Models that are over-parameterised (over-fitted) would tend to give small residuals for observations included in the model-fitting but large residuals for observations that are excluded.
%====================================================================%
Residuals plots

lme allows to plot the residuals in the following ways:

\begin{framed}
\begin{verbatim}
res_lme=residuals(model_lme)
plot(res_lme)
qqnorm(res_lme)
qqline(res_lme)
plot(model_lme)
\end{verbatim}
\end{framed}

%==============================================================%

CookD {predictmeans}	R Documentation
library(predictmeans)
CookD(model, group=NULL, plot=TRUE, idn=3, newwd=TRUE)

%======================================================================%

s a list, but does not have components 'x' and 'y'
> 
> fm1 <- lmList(dat)
> fm2 <- lme(dat)
> plot(compareFits(coef(fm1), coef(fm2)))
> plot(compareFits(JS.roy1,JS.roy2))



%===================================================================================================%
 
The LME for the ith subject is written as:
 
yi=0+1xi1+2xi2+b0+b1izi1+b2izi2+i 
 
 yi is the response variable.
 
The intercept and two methods are fixed effects, with  coefficients.
The two methods are also random effects, with b coefficients.
 
%===================================================================================================%

A multilevel mode is used to describe the blood pressure yijk measured.
 
yijk=j+bi+bi,k+ijk
 
Fixed Effects 
j describes the fixed effect of the jth method.
 
Random effects
bi describes the random effect of subject i
bi,k describes the kth replicate within subject effect 
 
Error
ijk residual.
 %===================================================================================================%
 
Laird Ware form for Single level grouping
 
yi=Xi+Zibi+i 
biN(0,)
iN(0,2I)
 
 
 
 
 
 

 
e2 is the within-subject variance of the 'established' method.
 
 
n2 is the within-subject variance of the 'new' method.
 
 
 
 
 
 
 
yN(X,V) 
 
uN(0,G)
 
eN(0,R)
  
V=Var(y)=ZGZT+ R
 
 
Blocki=Between Subj. Var.+Within Subj. Var. 
 
 
WGVC =ViCiVi
 
Vi describes the variance of the within-group errors 
Ci describes the correlation of the within-group errors 
 
i 
