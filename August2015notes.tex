 s a list, but does not have components 'x' and 'y'
> 
> fm1 <- lmList(dat)
> fm2 <- lme(dat)
> plot(compareFits(coef(fm1), coef(fm2)))
> plot(compareFits(JS.roy1,JS.roy2))

%===================================================================================================%
> shapiro.test(resid(JS.roy1)[1:255])

        Shapiro-Wilk normality test

data:  resid(JS.roy1)[1:255]
W = 0.9931, p-value = 0.2852

> shapiro.test(resid(JS.roy1)[256:510])

        Shapiro-Wilk normality test

data:  resid(JS.roy1)[256:510]
W = 0.9395, p-value = 9.503e-09

%===================================================================================================%
 
The LME for the ith subject is written as:
 
yi=0+1xi1+2xi2+b0+b1izi1+b2izi2+i 
 
 yi is the response variable.
 
The intercept and two methods are fixed effects, with  coefficients.
The two methods are also random effects, with b coefficients.
 
%===================================================================================================%

A multilevel mode is used to describe the blood pressure yijk measured.
 
yijk=j+bi+bi,k+ijk
 
Fixed Effects 
j describes the fixed effect of the jth method.
 
Random effects
bi describes the random effect of subject i
bi,k describes the kth replicate within subject effect 
 
Error
ijk residual.
 %===================================================================================================%
 
Laird Ware form for Single level grouping
 
yi=Xi+Zibi+i 
biN(0,)
iN(0,2I)
 
 
 
 
 
 

 
e2 is the within-subject variance of the 'established' method.
 
 
n2 is the within-subject variance of the 'new' method.
 
 
 
 
 
 
 
yN(X,V) 
 
uN(0,G)
 
eN(0,R)
  
V=Var(y)=ZGZT+ R
 
 
Blocki=Between Subj. Var.+Within Subj. Var. 
 
 
WGVC =ViCiVi
 
Vi describes the variance of the within-group errors 
Ci describes the correlation of the within-group errors 
 
i 
