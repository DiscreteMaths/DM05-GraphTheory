Bipartite graph
%- https://en.wikibooks.org/wiki/Graph_Theory/Definitions#Connectivity
%-------------------%
A graph ${\displaystyle G=(V,E)}$ is bipartite if the vertex set ${\displaystyle V}$ 
 can be partitioned into two disjoint subsets such that for every edge 
${\displaystyle (u,v)\in E}$,  
u {\displaystyle u} 
 and 
v {\displaystyle v} 
 are in opposite subsets. 
Additionally, a graph is bipartite if and only if it contains no odd cycles. 
%-------------------%
\subsection*{Proof: }
Let 
G = ( V , E ) {\displaystyle G=(V,E)} 
 be an arbitrary and fixed graph with an odd cycle.
G {\displaystyle G} 
 is bipartite 
⟹ G {\displaystyle \Longrightarrow G} 
 contains no odd cycles 
Assume for the sake of contradiction that 
G {\displaystyle G} 
 contains an odd cycle.
Let 
G 1 , G 2 {\displaystyle G_{1},G_{2}} 
 be two disjoint subsets that form a bipartition on 
G {\displaystyle G} 
.
Choose an odd cycle 
C {\displaystyle C} 
 in 
G {\displaystyle G} 
and, without loss of generality, choose some node 
v 1 ∈ C {\displaystyle v_{1}\in C} 
 from 
G 1 {\displaystyle G_{1}} 
. Traverse 
C {\displaystyle C} 
 starting from 
v 1 {\displaystyle v_{1}} 
, noting that each node is in the different subset 
G 1 , G 2 {\displaystyle G_{1},G_{2}} 
 from its neighbors by definition of a bipartite graph.
The last node in this traversal, 
v n {\displaystyle v_{n}} 
, must be in 
G 1 {\displaystyle G_{1}} 
, as 
C {\displaystyle C} 
 is an odd cycle. However, 
v n {\displaystyle v_{n}} 
 is adjacent to 
v 1 {\displaystyle v_{1}} 
 by definition of a cycle, which is a contradiction as 
G 1 {\displaystyle G_{1}} 
 contains two adjacent nodes, so 
G {\displaystyle G} 
 cannot contain an odd cycle.
As we have looked at an arbitrary 
C {\displaystyle C} 
 and chose 
v {\displaystyle v} 
 without loss of generality from 
G 1 {\displaystyle G_{1}} 
, we have proven the implication.
G {\displaystyle G} 
 contains no odd cycles 
⟹ G {\displaystyle \Longrightarrow G} 
 is bipartite 
We present an algorithm to sort the nodes into two disjoint sets. 
Fix an arbitrary node 
v ∈ V {\displaystyle v\in V} 
.
For each node 
x {\displaystyle x} 
 in the graph, calculate the shortest path from 
x {\displaystyle x} 
 to 
v {\displaystyle v} 
.
Let 
A , B {\displaystyle A,B} 
 be the sets of nodes such that the shortest path from 
x {\displaystyle x} 
 to 
v {\displaystyle v} 
 are even and odd respectively.
We look without loss of generality at 
A {\displaystyle A} 
 to prove this algorithm's correctness.
Assume for the sake of contradiction two nodes 
v 1 , v 2 ∈ A {\displaystyle v_{1},v_{2}\in A} 
 are adjacent. Let 
P 1 , P 2 {\displaystyle P_{1},P_{2}} 
 represent the shortest paths from 
v {\displaystyle v} 
 to 
v 1 , v 2 {\displaystyle v_{1},v_{2}} 
 respectively. 
If 
P 1 {\displaystyle P_{1}} 
 and 
P 2 {\displaystyle P_{2}} 
 do not intersect, 
P 1 , ( v 1 , v 2 ) , P 2 {\displaystyle P_{1},(v_{1},v_{2}),P_{2}} 
 forms a cycle of odd length as 
P 1 {\displaystyle P_{1}} 
 and 
P 2 {\displaystyle P_{2}} 
 trivially have the same parity, which is a contradiction.
If 
P 1 {\displaystyle P_{1}} 
 and 
P 2 {\displaystyle P_{2}} 
 intersect, we choose the intersection 
v ′ {\displaystyle v'} 
 closest to 
v 1 {\displaystyle v_{1}} 
. Note 
v ′ {\displaystyle v'} 
 cannot be either 
v 1 {\displaystyle v_{1}} 
 or 
v 2 {\displaystyle v_{2}} 
, as the shortest paths from 
v 1 {\displaystyle v_{1}} 
 to 
v 2 {\displaystyle v_{2}} 
 would not have the same parity. Observe that the distance from 
v ′ {\displaystyle v'} 
 to 
v 1 {\displaystyle v_{1}} 
 and 
v 2 {\displaystyle v_{2}} 
 must be the same length, otherwise the shortest path to either 
v 1 , v 2 {\displaystyle v_{1},v_{2}} 
 would travel through the other node. However, this is a contradiction as 
( v ′ , v 1 ) , ( v 1 , v 2 ) , ( v ′ , v 2 ) {\displaystyle (v',v_{1}),(v_{1},v_{2}),(v',v_{2})} 
 would be an odd cycle.
Therefore, no two nodes in 
A {\displaystyle A} 
 can be adjacent, and as we looked at 
A {\displaystyle A} 
 without loss of generality, we have proven the algorithm correct and therefore the reverse implication.

%-------------------%
\section{Complete Bipartite Graph}

The complete bipartite graph 
K m , n {\displaystyle K_{m,n}} 
 is the graph composed of two disjoint subsets 
A , B {\displaystyle A,B} 
 of cardinality 
m , n {\displaystyle m,n} 
 respectively, such that 
K m , n {\displaystyle K_{m,n}} 
 contains an edge between each node in 
A {\displaystyle A} 
 and every node in 
B {\displaystyle B} 
. 
