
%======================================================%
% Page 276

14.2 Representation of a postive-definitematrix ; the pdMat Class.

Positivedefinite classes are represented in the nlme package by objects inheriting from the \texttt{pdMat} class.

\begin{description}
\item[\texttt{pdIdent}] A mutliple of the Identity Matrix
\item[\texttt{CompSymm}] Compound Symmetry
\item[\texttt{pdNatural}] 
\item[\texttt{pdSymm}]
\item[\texttt{pdBlocked}]
\item[\texttt{odDiag}] A diagonal matrix
\end{description}


The constructor Function, used to create or to modify objects that inherit from a particular class, is named after the corresponding class.

For example \texttt{pdDiag()} function creates an object of the class \texttt{pdDiag}.

The \texttt{pdMat} constructors are primarily used in the specification of the random effects structure of an LME model, with the help of the random argument of the model-fitting function \texttt{lme()}.

%======================================================%
% PaGE 277

Arguments of the constructor function

\texttt{value},\texttt{form},\texttt{name} and \texttt{data}.

%======================================================%
%Page 283 - Random effects structure representation - the \texttt{reStruct} class


%======================================================%
% Page 284
Constructor function for the \texttt{Restruct}class.

The function \texttt{Restruct} is a constructor function for an object of class \texttt{reStruct}.
The arguments for the function are \texttt{object},\texttt{pdClass},\texttt{REML}, \texttt{data},\texttt{x},\texttt{Sigma},\texttt{reEstimates} and \texttt{verbose}.

%======================================================%
% Page 286
% 14.3.2
\subsection*{Inspecting and Modifying Objects of Class reStruct}

\texttt{isInitialized()}
\texttt{formula()}
\texttt{getGroupsFormula()}.
\section{Linear Mixed Effects Models}
\subsection{Extracting Information from a Model-Fit Object of class lme.}
% Page 293-295 Galecki

%--------------------------------------%
% Page 276
% 14.2 Representation of a positive definite matrix class - the "pdMat" class

\begin{description}
\item[\texttt{pdIdent}]: a multiple of identity
\item[\texttt{pdDiag}]:  a diagonal ,

\item[\texttt{pdCompSymm}]
\item[\texttt{pdLogChol}]
\item[\texttt{pdSymm}]
\item[\texttt{pdNatural}]
\item[\texttt{pdBlocked}]
\end{description}
%-------------------------------------%
% Page 283
% 14.3 Random Effects Structure Representation - the "restruct" class./
%      14.3.1 Constructor Function for the "reStruct" class
%      14.3.2 Inspecting an modifying Objects of the "reStruct" class 

%-------------------------------------%
% Page 292
% 14.5 Using the function "lme" to specify and fit LME models

%-------------------------------------%
% Page 293
% 14.6 Extracting Information from a Model-Fit object of "nlme" class 


%-------------------------------------%

% Page 297
% 14.7 Tests of Hypotheses about model parameters


%-------------------------------------------------------------%
% Page 293 

Galecki discusses several method to extract from a model fit of class lme.
For an R object lme.ft, which contains the results of fitting a single level LME.
By default, level=0.95 is ised.
%-------------------------------------------------------------%
% Page 294 
\begin{itemize}
\item ML Value logLik(lme.fit,REML=FALSE)
\item REML Value logLik(lme.fit,REML=FALSE)
\item AIC(lme.fit) AIC(lme.fit)
\item BIC(lme.fit) BIC(lme.fit)
\end{itemize}

%-------------------------------------------------------------%
%Page 295 
\begin{description}
\item[interval()]: computes confidence intervals for the model parameters, with named components.
\item[fixef()]: fixed effects
\item[ranef()]: by applying the function ranef() to a lme class model-fit object, the estimated random
effects are displayed. By default, the effects of all levels of the groupings are displayed.
\item[coef()]:  (help: ?coef.lme)
\item[corStruct()]: 
\item[getVarCov()]:
\item[VarCorr()]
\end{description}

%-------------------------------------------------------------%
%PAge 296
predict() : new data


\subsection{Tests of Hypothesis about Model Parameters}
MArginal approach tests, the argument type marginal should be used.



anova() is applied to two or more objects of the class lme it provides LR statistics, calculated on consecutive pairs of objects.



%Page 299 The function simulate.


%------------------------------------------------------------------
exactRLRM
\end{document}
