\documentclass{article}

\usepackage{amsmath}
\usepackage{amssymb}
\usepackage{graphicx}

\begin{document}

%http://webwhompers.com/graph-theory.html
\section{Graph Theory}
%-------------------------------- %
Example 5.5 A graph can be used in chemistry to model the structure of a molecule. In this
model, the atoms of the elements composing the molecule are taken as the vertices of the graph
and the edges represent the chemical bonds between pairs of atoms. So for example, a graph model
of the water molecule $H_2O$ would have three vertices and two edges. The degree of a vertex is the
valency of the atom it represents. 
%-------------------------------- %
Since an early use of graph theory (by Arthur Cayley, in the
mid 19th century) was to count the number of different isomers of some of the hydrocarbons, the
term valency was initially used instead of degree of a vertex, and you will still find this usuage in
some text books. 
%-------------------------------- %
We next investigate the connection between the number of edges in a graph and the degrees of
the vertices. If, for each of the graphs shown Figures 5.1, 5.2 and 5.3, you sum the degrees of the
vertices and compare this sum with the number of edges of the graph, you should find that the
following result holds.
%-------------------------------- %
Result 5.3 
\subsection*{Degrees}
Let G be a graph. Then the sum of the degrees of the vertices of G is equal to twice
the number of edges of G.
%-------------------------------- %
Proof. Let e be any edge of G and suppose that the endpoints of e are u and v. Then e contributes
1 both to deg(u) and to deg(v). (If e is a loop, then u = v and e contributes 2 to deg(u)). Thus
every edge contributes 2 to the sum of the degrees of the vertices, giving the result. 2
% 5.4.1 
%-------------------------------- %
Example 5.6 A graph can be used to model acquaintanceship between pairs of people in a group.
In this model, each person in the group is represented by a vertex. Two vertices in the graph are
joined by an edge if and only if the two people they represent know one another. 
%-------------------------------- %
Thus the number of edges in this graph gives the number of pairs of people in the group who know one another.
(Note that in order to be useful, acquaintance graphs are simple graphs; that is, we do not record
the obvious fact that each person knows himself or herself by putting a loop at each vertex.)
Suppose that there are ten people in the group and we ask each of them how many other people
in the group they are acquainted with. Suppose they give us the following numbers:
\[2, 5, 6, 6, 4, 3, 9, 1, 7, 5\].
%-------------------------------- %
We cannot construct the acquaintance graph from this information alone, because we do not know
which other people in the group each person knows.

However, we can calculate the number of
edges in the graph from just the information given. This is because the number of acquaintances
of a given member of the group gives us the degree of the vertex representing her or him. Summing
the sequence of numbers given above, we obtain
\[2 + 5 + 6 + 6 + 4 + 3 + 9 + 1 + 7 + 5 = 48.\]

Hence, using Result 5.3, the number of different pairs of acquaintances in the group is 48/2 = 24.

%-------------------------------- %
\subsection*{Degree Sequence}
Definition 5.4 \textbf{The degree sequence }of a graph G is the sequence formed from the degrees of its
vertices, usually arranged in descending order of size.

%-------------------------------- %
Example 5.7 The degree sequence for the acquaintance graph described in Example 5.6 is:
\[9, 7, 6, 6, 5, 5, 4, 3, 2, 1.\]

%-------------------------------- %
\subsection*{ Paths }

Definition 5.10 A path is an alternating sequence of vertices and edges of the form 
                                  \[v_1 e_1v_2 e_2v_3 \ldots e_{k-1}v_k,\] 
where ei is the edge joining vi to vi+1, and all the vertices and all the edges are distinct. Note that in a simple graph, there is at most one edge joining any pair of vertices and so a path can be specified just by a sequence of distinct vertices.
%----------------------------------------%

Definition 5.11 The \textbf{length} of a path is the number of edges in it.

A common mistake is to take the number of vertices in a path as its length instead of the number of edges.


%----------------------------------------%
%5.4.3 
\subsection*{Connectivity}

Definition 5.13 Two vertices u, v in a graph G are said to be connected if there is a 
path in G from u to v. The graph G is said to be connected if every pair of vertices are 
connected; otherwise, G is said to be disconnected.

Example 5.19 

You can visualize a connected graph as being one that is "\textit{\textbf{all in one piece}}''. The separate connected parts of a disconnected graph are called its connected components. Thus the graph shown in Figure 5.3 has 2 connected components; every connected graph has just one connected component, the graph itself.

	
%----------------------------------------%
% 5.5 
\subsection*{Isomorphism of graphs}

Learning objectives
When you have completed this section, you should be able to:

state the definition of graph isomorphism;
decide whether two given graphs on n vertices (for small values of n) are isomorphic and justify your decision.


%----------------------------------------%
%5.6.1
\subsection*{ The adjacency matrix of a graph}

While graphs are useful for modelling many different applications, the diagrammatic representations we have been discussing are not easily understood by computers. The most common way to represent a graph within a computer is as a square array of numbers, each number representing the number of edges joining a pair of vertices.

Definition 5.17 Suppose that G is a graph with n vertices, numbered $1, 2, \ldots , n$. Then the adjacency matrix A(G) of G is a square $n \times n$ array, with rows and columns numbered $1, 2, \ldots , n$ such that the entry in row i and column j is the number of edges joining vertex i to vertex j.

% Example 5.23 
%----------------------------------------%

\subsection*{Square Matrices}
Definition 5.18 The diagonal line of cells from the top left hand cell to the bottom right hand cell of a square matrix is called the main diagonal. A square matrix is called symmetric if the entries in row i are the same as the entries in column i, for $i = 1, 2, \ldots , n$.

The adjacency matrix of a graph is always symmetric because the number of edges joining vertex i to vertex j is equal to the number of edges joining vertex j to vertex i. Geometrically, the symmetry of the matrix is about the main diagonal. Note that the entries on the main diagonal record the number of loops at each vertex.

%==============================================================================================%
%-https://en.wikibooks.org/wiki/Graph_Theory/Definitions#Distance_in_a_Graph

Walk, closed walk, circuit and cycle[edit]
A u-v walk is defined as a sequence of vertices starting at u and ending at v, where consecutive vertices in the sequence are adjacent vertices in the graph

A drawing of a labelled graph on 6 vertices and 7 edges.
A closed walk is a walk in which the first and last vertices are the same
A u-v trail is a u-v walk, where no edge is repeated (each edge is used at most once)
A circuit or closed trail is a trail in which the first and last vertices are the same
A u-v path is a u-v walk, where no vertex is repeated (each vertex is used at most once)
A cycle is a closed path in which the first and last vertices are the same
For example, in the image to the right, (6,4,5,1) is a walk. While (4,5,2,3,4), and (1,5,2,1) are cycles.


%----------------------------------------%
%5.6.2 
\subsection*{Adjacency lists}

In some large graphs, each vertex may only be adjacent to relatively few other vertices. In this case, a high proportion of the entries in the adjacency matrix will be zero, which is wasteful of storage space. An alternative method of storing a simple graph is to use a set of adjacency lists. First, we list the vertices of the graph; then, after each vertex, we put a colon and list the vertices that it is adjacent to. The procedure is illustrated in the example below.
%----------------------------------------%
\end{document}
