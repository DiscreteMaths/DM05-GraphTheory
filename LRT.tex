
\newpage
\section{Likelihood ratio tests}
Likelihood ratio tests are  a class o tests based on the comparison of the values of the likelihood functions of two
candidate models. LRTs can be used to test hypotheses about covariance parameters or fixed effects parameters in the context
of LMEs.

The test statistic for the LRT is the difference of the log-likelihood functions, multiplied by $-2$.
The probability distribution of the test statistic is approximated by the $\chi^2$ distribution with ($\nu_{1} - \nu_{2}$) degrees of freedom, where $\nu_{1}$  and $\nu_{2}$ are the degrees of freedom of models 1 and 2 respectively.

The score function $S(\theta)$ is the derivative of the log likelihood with respect to $\theta$,

\[
S(\theta) = \frac{\partial}{\partial \theta}\emph{l}(\theta),
\]

and the maximum likelihood estimate is the solution to the score equation
\[
S(\theta) = 0.
\]
The Fisher information $I(\theta)$, which is defined as
\[
I(\theta) = - \frac{\partial^2}{\partial \theta^2}\emph{l}(\theta),
\]
give rise to the observed Fisher information ($I(\hat{\theta})$) and the expected Fisher information ($\mathcal{I}(\theta)$).

