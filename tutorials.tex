
\item  Consider the pair of graphs below. Are these graphs isomorphic? Justify your answer.


%GraphTheory-Isomorph-1
%==================%
\subsection{Question 6} A company operates tut express .-ortelt service boo.% seven cities. 
The number of othcr cities to which each city is direct;• linked by a coach is given in the-following table. 
Cit9 
C2 t, c c5 C6 C7 
Nurnin,,r of co►tro:clions. 3 2 3 • 
4 
4 
1 
\begin{enumerate}[(i)]
\item Describe how such it communications netw•it can be modelled by a gray d,. saying what the vertices represent and a rule for determining when two verticcE are adjacent. 
\item  Calculate how many pairs of cities have P. direct coach link between giving a brief explanation of your method. Whet is meant by saying that a graph is • rtpl,.? Say why a graph model of this communications. ner•:,-)rk would be sin...pita.. 
(r) 
\item  Is it POSSibiO ti) construct a graph with degr,-t s,,quenee 4.4. 4, 3. 3. 2. 1? Either construct an example of such a graph. or by it is not possible to do so 
\end{enumerate}
\subsection{Question 6}
 The following adjacency matrix shows several European countries and an entry of 1 indicates the countries concerned share a common border, whereas a zero entry indicates they do not. 
Austria Belgium France Germany Italy 
( 
Austria Belgium France Germany Italy o 1 0 0 1 1 0 1 1 1 o 1 0 1 1 0 1 1 0 1 1 1 1 1 0 
(i) Write down the countries which share a border with Germany. (ii) Is this matrix symmetric or not? Give an example to show what this means. (iii) Draw the graph. G, associated with this matrix. (iv) Explain how the number of edges of the graph can be calculated from the entries in the matrix and find this number. (v) Draw another graph, H, which has 5 vertices and the same degree sequence as G but is not isomorphic to it. Give a reason why G and H are not isomorphic. [6] 


\subsection{Question 5} (a) Let G be a simple graph with vertex set V (G) = {vi, v2, v3, v4, v5} and adjacency lists as follows: 
VI • V2 V3 V4 V2 Vl v3 v4 vs 
vi v.1 v4 vi v2 v3. v5 v2 
(i) List the degree sequence of G. (ii) Draw the graph of G. (iii) Find two distinct paths of length 3, starting at va and ending at v4. (iv) Find a 4 cycle in G. 
[6] 
(b) Let Ky, be the simple graph with vertices vZ, v2, v3, ..., i.'3 in which each vertex is joined to every other vertex by an edge. 
(I) Draw K6. (ii) Determine the number of edges of K6. (iii) Determine the number of paths from vi to v2 of length two. (iv) Find an expression in terms of n for the number of paths from vi to v2 of length two in kn. [5] 
(c) Draw two different (that is non-isomorphic) connected graphs each having the degree sequence 3,3.2,1, 1,1, 1. Give one reason why the graphs you have drawn are not isomorphic. 131 
(a) Let G be a simple graph. Explain why the sum of the degrees of the vertices of G is twice the number of its edges. (b) Justifying your answer, say why it is not possible to construct a simple graph G with degree sequence 

[2] 
[2] 
(c) Justifying your answer, say whether it is possible to construct a simple graph with degree sequence 3,3,3,3,3,3. [2] 

\newpage




%-------------------------------------------------------- %


Graph theory is the study of points and lines. In particular, it involves the ways in which sets of points, called \textit{\textbf{vertices}}, can be connected by lines or arcs, called \textit{\textbf{edges}}.

Graphs are classified according to their complexity, the number of edges allowed between any two vertices, and whether or not directions (for example, up or down) are assigned to edges. 

%Various sets of rules result in specific properties that can be stated as theorems.

%------------------------------------------------------- %
\section*{Adjacency}
\begin{itemize}

\item[(a)] An \textit{\textbf{adjacency list}} representation of a graph is a collection of unordered lists, one for each vertex in the graph. Each list describes the set of neighbors of its vertex.

\item[(b)] An \textit{\textbf{adjacency matrix}} is a means of representing which vertices (or nodes) of a graph are adjacent to which other vertices. Another matrix representation for a graph is the incidence matrix.

\item[(c)]

\end{itemize}
%-------------------------------------------------------- %

%-------------------------------------------------------- %
\newpage


\item[5.d]
Degree sequence 4,3,2,2,2
Degree sequence 4,3,3,2,2

\item[5.e]
K8 has degree sequence 7,7,7,7,7,7,7,7 so every vertex has degree 7.
\end{itemize}

\end{document}

A graph consists of a finite set of vertices $V$ and a finite set of Edges $E$.
%---------------------------------------------%
The degree of a vertex $v_1$ denoted by $deg(v)$ is the number of edges incident with $v$.
A graph in which every vertex has the same degree r is called \textbf{\emph{r-regular}}.
% page 73


%---------------------------------------------%
\newpage

A path is an alternating sequences of vertices and edges of the form
$v_1e_1v_2e_2 \ldots e_{k-1}v_k$

The length of a path is the number of edges in it.

\subsection{Isomorphism of a Graph}

%---------------------------------------------%
%5.3

\end{document}

%---------------------------------------------%

