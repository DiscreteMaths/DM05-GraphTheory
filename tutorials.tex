
\item  Consider the pair of graphs below. Are these graphs isomorphic? Justify your answer.


%GraphTheory-Isomorph-1
%==================%
\subsection{Question 6} A company operates tut express .-ortelt service boo.% seven cities. 
The number of othcr cities to which each city is direct;• linked by a coach is given in the-following table. 
Cit9 
C2 t, c c5 C6 C7 
Nurnin,,r of co►tro:clions. 3 2 3 • 
4 
4 
1 
\begin{enumerate}[(i)]
\item Describe how such it communications netw•it can be modelled by a gray d,. saying what the vertices represent and a rule for determining when two verticcE are adjacent. 
\item  Calculate how many pairs of cities have P. direct coach link between giving a brief explanation of your method. Whet is meant by saying that a graph is • rtpl,.? Say why a graph model of this communications. ner•:,-)rk would be sin...pita.. 
(r) 
\item  Is it POSSibiO ti) construct a graph with degr,-t s,,quenee 4.4. 4, 3. 3. 2. 1? Either construct an example of such a graph. or by it is not possible to do so 
\end{enumerate}
\subsection{Question 6}
 The following adjacency matrix shows several European countries and an entry of 1 indicates the countries concerned share a common border, whereas a zero entry indicates they do not. 
Austria Belgium France Germany Italy 
( 
Austria Belgium France Germany Italy o 1 0 0 1 1 0 1 1 1 o 1 0 1 1 0 1 1 0 1 1 1 1 1 0 
(i) Write down the countries which share a border with Germany. (ii) Is this matrix symmetric or not? Give an example to show what this means. (iii) Draw the graph. G, associated with this matrix. (iv) Explain how the number of edges of the graph can be calculated from the entries in the matrix and find this number. (v) Draw another graph, H, which has 5 vertices and the same degree sequence as G but is not isomorphic to it. Give a reason why G and H are not isomorphic. [6] 








\item[5.d]
Degree sequence 4,3,2,2,2
Degree sequence 4,3,3,2,2

\item[5.e]
K8 has degree sequence 7,7,7,7,7,7,7,7 so every vertex has degree 7.
\end{itemize}

\end{document}

A graph consists of a finite set of vertices $V$ and a finite set of Edges $E$.
%---------------------------------------------%
The degree of a vertex $v_1$ denoted by $deg(v)$ is the number of edges incident with $v$.
A graph in which every vertex has the same degree r is called \textbf{\emph{r-regular}}.
% page 73


%---------------------------------------------%
\newpage

A path is an alternating sequences of vertices and edges of the form
$v_1e_1v_2e_2 \ldots e_{k-1}v_k$

The length of a path is the number of edges in it.

\subsection{Isomorphism of a Graph}

%---------------------------------------------%
%5.3

\end{document}

%---------------------------------------------%

