\documentclass[Main.tex]{subfiles}
\begin{document}
	%====================%\\
	
	As a complement to this, one can also consider how to properly employ the $R^2$ measure, in the context of Methoc Comparison Studies, further to the work by Edwards et al, namely ``An $R^2$ statistic for fixed effects in the linear mixed model".
%================================================= %
\newpage
\begin{framed}

	\begin{quote}
		\textbf{Abstract for ``An $R^2$ statistic for fixed effects in the linear mixed model"}
		Statisticians most often use the linear mixed model to analyze Gaussian longitudinal data. 
		
		The value and familiarity of the R2 statistic in the linear univariate model naturally creates great interest in extending it to the linear mixed model. We define and describe how to compute a model R2 statistic for the linear mixed model by using only a single model. 
		
		The proposed R2 statistic measures multivariate association between the repeated outcomes and the fixed effects in the linear mixed model. The R2 statistic arises as a 1–1 function of an appropriate F statistic for testing all fixed effects (except typically the intercept) in a full model. 
		
		The statistic compares the full model with a null model with all fixed effects deleted (except typically the intercept) while retaining exactly the same covariance structure. 
		
		Furthermore, the R2 statistic leads immediately to a natural definition of a partial R2 statistic. A mixed model in which ethnicity gives a very small p-value as a longitudinal predictor of blood pressure (BP) compellingly illustrates the value of the statistic. 
		
		In sharp contrast to the extreme p-value, a very small $R^2$ , a measure of statistical and scientific importance, indicates that ethnicity has an almost negligible association with the repeated BP outcomes for the study.
	\end{quote}
\end{framed}

\end{document}