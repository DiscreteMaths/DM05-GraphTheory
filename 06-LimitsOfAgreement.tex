\documentclass[Main.tex]{subfiles}
\begin{document}
	%========================================================================== %
\section{Limits of agreement in LME models}

Limits of agreement are used extensively for assessing agreement, because they are intuitive and easy to use.
Necessaire their prevalence in literature has meant that they are now the best known measurement for agreement, and therefore any newer methodology would benefit by making reference to them.

\citet{BXC2008} uses LME models to determine the limits of agreement. Between-subject variation for method $m$ is given by $d^2_{m}$ and within-subject variation is given by $\lambda^2_{m}$.  \citet{BXC2008} remarks that for two methods $A$ and $B$, separate values of $d^2_{A}$ and $d^2_{B}$ cannot be estimated, only their average. Hence the assumption that $d_{x}= d_{y}= d$ is necessary. The between-subject variability $\boldsymbol{D}$ and within-subject variability $\boldsymbol{\Lambda}$ can be presented in matrix form,\[
\boldsymbol{D} = \left(%
\begin{array}{cc}
   d^2_{A}& 0 \\
  0 & d^2_{B} \\
\end{array}%
\right)=\left(%
\begin{array}{cc}
   d^2& 0 \\
  0 & d^2\\
\end{array}%
\right),
\hspace{1.5cm}
\boldsymbol{\Lambda} = \left(%
\begin{array}{cc}
   \lambda^2_{A}& 0 \\
  0 & \lambda^2_{B} \\
\end{array}%
\right).
\]

The variance for method $m$ is $d^2_{m}+\lambda^2_{m}$. Limits of agreement are determined using the standard deviation of the case-wise differences between the sets of measurements by two methods $A$ and $B$, given by
\begin{equation}
\mbox{var} (y_{A}-y_{B}) = 2d^2 + \lambda^2_{A}+ \lambda^2_{B}.
\end{equation}
Importantly the covariance terms in both variability matrices are zero, and no covariance component is present.

\citet{BXC2008} presents a data set `fat', which is a comparison of measurements of subcutaneous fat
by two observers at the Steno Diabetes Center, Copenhagen. Measurements are in millimeters
(mm). Each person is measured three times by each observer. The observations are considered to be `true' replicates.

A linear mixed effects model is formulated, and implementation through several software packages is demonstrated.
All of the necessary terms are presented in the computer output. The limits of agreement are therefore,
\begin{equation}
0.0449  \pm 1.96 \times  \sqrt{2 \times 0.0596^2 + 0.0772^2 + 0.0724^2} = (-0.220,  0.309).
\end{equation}

\citet{roy} has demonstrated a methodology whereby $d^2_{A}$ and $d^2_{B}$ can be estimated separately. Also covariance terms are present in both $\boldsymbol{D}$ and $\boldsymbol{\Lambda}$. Using Roy's methodology, the variance of the differences is
\begin{equation}
\mbox{var} (y_{iA}-y_{iB})= d^2_{A} + \lambda^2_{B} + d^2_{A} + \lambda^2_{B} - 2(d_{AB} + \lambda_{AB})
\end{equation}
All of these terms are given or determinable in computer output.
The limits of agreement can therefore be evaluated using
\begin{equation}
\bar{y_{A}}-\bar{y_{B}} \pm 1.96 \times \sqrt{ \sigma^2_{A} + \sigma^2_{B}  - 2(\sigma_{AB})}.
\end{equation}

For Carstensen's `fat' data, the limits of agreement computed using Roy's
method are consistent with the estimates given by \citet{BXC2008}; $0.044884  \pm 1.96 \times  0.1373979 = (-0.224,  0.314).$


%------------------------------------------------------------------------------------%
\subsection{Linked replicates}

\citet{BXC2008} proposes the addition of an random effects term to their model when the replicates are linked. This term is used to describe the `item by replicate' interaction, which is independent of the methods. This interaction is a source of variability independent of the methods. Therefore failure to account for it will result in variability being wrongly attributed to the methods.

\citet{BXC2008} introduces a second data set; the oximetry study. This study done at the Royal Children’s Hospital in
Melbourne to assess the agreement between co-oximetry and pulse oximetry in small babies.

In most cases, measurements were taken by both method at three different times. In some cases there are either one or two pairs of measurements, hence the data is unbalanced. \citet{BXC2008} describes many of the children as being very sick, and with very low oxygen saturations levels. Therefore it must be assumed that a biological change can occur in interim periods, and measurements are not true replicates.

\citet{BXC2008} demonstrate the necessity of accounting for linked replicated by comparing the limits of agreement from the `oximetry' data set using a model with the additional term, and one without. When the interaction is accounted for the limits of agreement are (-9.62,14.56). When the interaction is not accounted for, the limits of agreement are (-11.88,16.83). It is shown that the failure to include this additional term results in an over-estimation of the standard deviations of differences.

Limits of agreement are determined using Roy's methodology, without adding any additional terms, are found to be consistent with the `interaction' model; $(-9.562, 14.504 )$. Roy's methodology assumes that replicates are linked. However, following Carstensen's example, an addition interaction term is added to the implementation of Roy's model to assess the effect, the limits of agreement estimates do not change. However there is a conspicuous difference in within-subject matrices of Roy's model and the modified model (denoted $1$ and $2$ respectively);
\begin{equation}
\hat{\boldsymbol{\Lambda}}_{1}= \left(\begin{array}{cc}
16.61 &	11.67\\
11.67 & 27.65 \end{array}\right) \qquad
\boldsymbol{\hat{\Lambda}}_{2}= \left( \begin{array}{cc}
7.55 & 2.60 \\
2.60 & 18.59 \end{array} \right).
\end{equation}

\noindent (The variance of the additional random effect in model $2$ is $3.01$.)

\citet{akaike} introduces the Akaike information criterion ($AIC$), a model
selection tool based on the likelihood function. Given a data set, candidate models
are ranked according to their AIC values, with the model having the lowest AIC being considered the best fit.Two candidate models can said to be equally good if there is a difference of less than $2$ in their AIC values.

The Akaike information criterion (AIC) for both models are $AIC_{1} = 2304.226$ and $AIC_{2} = 2306.226$ , indicating little difference in models. The AIC values for the Carstensen `unlinked' and `linked' models are $1994.66$ and $1955.48$ respectively, indicating an improvement by adding the interaction term.

The $\boldsymbol{\hat{\Lambda}}$ matrices are informative as to the difference between Carstensen's unlinked and linked models. For the oximetry data, the covariance terms (given above as 11.67 and 2.6 respectively ) are of similar magnitudes to the variance terms. Conversely for the `fat' data the covariance term ($-0.00032$) is negligible. When the interaction term is added to the model, the covariance term remains negligible. (For the `fat' data, the difference in AIC values is also approximately $2$).

To conclude, Carstensen's models provided a rigorous way to determine limits of agreement, but don't provide for the computation of $\boldsymbol{\hat{D}}$ and $\boldsymbol{\hat{\Lambda}}$. Therefore the test's proposed by \citet{roy} can not be implemented. Conversely, accurate limits of agreement as determined by Carstensen's model may also be found using Roy's method. Addition of the interaction term erodes the capability of Roy's methodology to compare candidate models, and therefore shall not be adopted.

Finally, to complement the blood pressure (i.e.`J vs S') method comparison from the previous section (i.e.`J vs S'), the limits of agreement are $15.62 \pm 1.96 \times 20.33 = (-24.22, 55.46)$.)

\end{document}