\documentclass[]{report}

\voffset=-1.5cm
\oddsidemargin=0.0cm
\textwidth = 480pt

\usepackage{framed}
\usepackage{subfiles}
\usepackage{graphics}
\usepackage{newlfont}
\usepackage{eurosym}
\usepackage{amsmath,amsthm,amsfonts}
\usepackage{amsmath}
\usepackage{color}
\usepackage{enumerate}
\usepackage{amssymb}
\usepackage{multicol}
\usepackage[dvipsnames]{xcolor}
\usepackage{graphicx}
\begin{document}

\section{Graph Theory - Types of Graphs}

\begin{itemize}
\item Spanning Subgraphs of G.

\item a vertex is said to be an \textbf{emph{ isolated vertex}} if it has a degree of zero.
\item a vertex is said to be an \textbf{emph{ end-vertex}} if it has a degree of one.
\item a vertex is said to be an \textbf{emph{ even vertex}} if it has a degree of an even number.
\item a vertex is said to be an \textbf{emph{ odd vertex}} if it has a degree of an odd number.


\item A graph is said to be \textbf{emph{k-regular}} if the degree of each vertex is $k$. 
\item Every Graph has an even number of odd vertices.
\item A cubic graph is a graph where every vertex has degree three.
\end{itemize}
\newpage
\section{Types of Graphs}

\subsection{Directed Graphs}
A directed graph is a graph where the arcs are one-directional, that is an arc between two nodes only indicates the possibility of moving from one node to the other, but not in the opposite direction. The arcs will usually be drawn as arrows to indicate the direction. An example of a potential use for a directed graph would be a graph which tracks possible states that a computer could be in; there may be a way for a computer in one state to reach a subsequent state, but no way to return from the second state to the first.

\subsection{Weighted graph}
A weighted graph is a graph where there is some 'cost' associated with each arc. Usually, a little number will appear directly adjacent to every arc to express this price. A typical use of a weighted graph would be planning routes between locations on a map where the 'cost' associated with the arc would be the distance between the two locations.

\subsection{Trees}
A tree is a special graph which is connected (every node can be reached from every other node by following one or more arcs) and for which the number of arcs is exactly one fewer than the number of nodes. A tree is usually drawn with one node (called the 'root node') at the top of the diagram, and then 'growing' downwards with arcs and nodes getting further and further from the root. In this way, nodes can be grouped in terms of their distance from the root. Every node (aside from the root) is referred to as the 'child' of the node to which it is connected and which is one step closer to the root. This node is also called the 'parent' node of the child. Every node has at most one parent but can have any number of children. Nodes without any children are commonly called 'leaf nodes'. A typical use of a tree would be a decision problem where the answer to a question determines the next question and set of possible answers.
\section{Planar Graphs}
A planar graph is a graph that can be drawn in the plane such that there are no edge crossings.

\subsection{Characterization}
The planar graphs can be characterized by a theorem first proven by Kazimierz Kuratowski in 1930, which states that the planar graphs are exactly those graphs G such that $K_5 \not \preceq G$  and $K_{3,3} \not \preceq G$ .



\subsection{Types of Graph}
\begin{itemize}
\item Bipartite Graphs
\item Path Graphs
\item Cycle Graphs
\item Null Graphs
\end{itemize}

\subsection{Bipartite Graphs}
\begin{itemize}
\item A bipartite graph is a graph whose vertex-set can be split into two sets in such a way that each edge of the graph joins a vertex in first set to a vertex in second set.
\end{itemize}


\subsection{Path Graphs}
\begin{itemize}
\item A path graph is a graph consisting of a single path. 
\item The path graph with n vertices is denoted by $P_n$.
\end{itemize}


\subsection{Cycle Graphs}
\begin{itemize}
\item A cycle graph is a graph consisting of a single cycle. 
\item The cycle graph with n vertices is denoted by $C_n$.
\end{itemize}


\subsection{Null Graphs}
\begin{itemize}
\item A null graphs is a graph containing no edges. \item The null graph with n vertices is denoted by $N_n$.
\end{itemize}



There are various types of graphs depending upon the number of vertices, number of edges, interconnectivity, and their overall structure. We will discuss only a certain few important types of graphs in this chapter.





\end{document}
