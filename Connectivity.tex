Connectivity[edit]
Whether or not it is possible to traverse a graph from one vertex to another is dependent on how connected a graph is. 
Definition of Connectedness[edit]
If there is a u-v path in G, then we say that u and v are connected
If there is a u-v path for every pair of vertices u and v in G, then we say that G is connected
Theorems on Connectedness[edit]
Theorem: Let G be a graph of order at least 3. If there are distinct vertices u and v in G so that G-u and G-v are both connected, then G is also connected. 
Proof: Let w be a G vertex, which is different from both u and v. We want to prove connectedness, i.e., that for every pair (x,y) of vertices in G, there is an x-y walk in G. We may consider three (partly overlapping) cases: 
If neither x nor y is u, then there is an x-y walk in G-u, and this also is an x-y walk in G.
If neither x nor y is v, then there is an x-y walk in G-v, and this also is an x-y walk in G.
If {x,y} = {u,v}, then employ the first two cases in order to see that there is a u-w walk and a w-v walk. Combining them indeed yields a u-v walk.
Vertex and Edge Connectivity[edit]
A graph G is called k-connected if for every 
S ⊆ V ( G ) {\displaystyle S\subseteq V(G)} 
, and 
| S | < k {\displaystyle |S|<k} 
, 
G − S {\displaystyle G-S} 
 is connected, and 
| G | > k {\displaystyle |G|>k} 
. 
Similarly, a graph G is called 
ℓ {\displaystyle \ell } 
 edge-connected if for every 
S ⊆ E ( G ) {\displaystyle S\subseteq E(G)} 
, and 
| S | < ℓ {\displaystyle |S|<\ell } 
, 
G − S {\displaystyle G-S} 
 is connected, and 
| G | > 1 {\displaystyle |G|>1} 
. 
The connectivity of G is the greatest k such that G is k-connected, and is denoted by 
κ ( G ) {\displaystyle \kappa (G)} 
. 
Relatedly, the edge-connectivity of G is the greatest 
ℓ {\displaystyle \ell } 
 such that G is 
ℓ {\displaystyle \ell } 
 edge-connected, and is denoted by 
λ ( G ) {\displaystyle \lambda (G)} 
. 
Theorems on Connectivity[edit]
Theorem: Let G be a k-connected graph. Then, 
∀ i ∈ N , 0 ≤ i ≤ k {\displaystyle \forall i\in \mathbb {N} ,0\leq i\leq k} 
, G is i-connected. 
Proof: Since G is k-connected, for all 
S ⊆ V ( G ) , | S | < k {\displaystyle S\subseteq V(G),|S|<k} 
, 
G − S {\displaystyle G-S} 
 is connected. Then, since 
i ≤ k {\displaystyle i\leq k} 
, for all 
S ⊆ V ( G ) , | S | < i ≤ k {\displaystyle S\subseteq V(G),|S|<i\leq k} 
, 
G − S {\displaystyle G-S} 
 is also connected. 

Theorem: Let G be a nontrivial graph. Then, 
λ ( G ) ≤ δ ( G ) {\displaystyle \lambda (G)\leq \delta (G)} 
. 
Proof: Take v a vertex of degree 
δ ( G ) {\displaystyle \delta (G)} 
 in G. Then, removing all edges in G that are incident with v disconnects v from the rest of the graph, provided that 
| G | > δ ( G ) + 1 {\displaystyle |G|>\delta (G)+1} 
. Therefore, G cannot be 
δ ( G ) {\displaystyle \delta (G)} 
 edge-connected, and so 
λ ( G ) ≤ δ ( G ) {\displaystyle \lambda (G)\leq \delta (G)} 
. 

Exercise: Connectivity
If G is 
ℓ {\displaystyle \ell } 
 edge-connected, show that G is also i edge-connected 
∀ i ∈ N , 0 ≤ i ≤ ℓ {\displaystyle \forall i\in \mathbb {N} ,0\leq i\leq \ell } 
.
