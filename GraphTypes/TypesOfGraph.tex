% https://www.tutorialspoint.com/graph_theory/graph_theory_trees.htm

\subsection{Null Graph}
A graph having no edges is called a Null Graph.

Example
Null Graph
In the above graph, there are three vertices named ‘a’, ‘b’, and ‘c’, but there are no edges among them. Hence it is a Null Graph.

\subsection{Trivial Graph}
A graph with only one vertex is called a Trivial Graph.

Example
Trivial
In the above shown graph, there is only one vertex ‘a’ with no other edges. Hence it is a Trivial graph.

Non-Directed Graph
A non-directed graph contains edges but the edges are not directed ones.

Example
Non-Directed
In this graph, ‘a’, ‘b’, ‘c’, ‘d’, ‘e’, ‘f’, ‘g’ are the vertices, and ‘ab’, ‘bc’, ‘cd’, ‘da’, ‘ag’, ‘gf’, ‘ef’ are the edges of the graph. Since it is a non-directed graph, the edges ‘ab’ and ‘ba’ are same. Similarly other edges also considered in the same way.

\subsection{Directed Graph}
In a directed graph, each edge has a direction.

Example
Directed Graph
In the above graph, we have seven vertices ‘a’, ‘b’, ‘c’, ‘d’, ‘e’, ‘f’, and ‘g’, and eight edges ‘ab’, ‘cb’, ‘dc’, ‘ad’, ‘ec’, ‘fe’, ‘gf’, and ‘ga’. As it is a directed graph, each edge bears an arrow mark that shows its direction. Note that in a directed graph, ‘ab’ is different from ‘ba’.

\subsection{Simple Graph}
A graph with no loops and no parallel edges is called a simple graph.

The maximum number of edges possible in a single graph with ‘n’ vertices is nC2 where nC2 = n(n – 1)/2.

The number of simple graphs possible with ‘n’ vertices = 2nc2 = 2n(n-1)/2.

Example
In the following graph, there are 3 vertices with 3 edges which is maximum excluding the parallel edges and loops. This can be proved by using the above formulae.

Simple Graph
The maximum number of edges with n=3 vertices −

nC2 = n(n–1)/2
   = 3(3–1)/2
   = 6/2
   = 3 edges
The maximum number of simple graphs with n=3 vertices −

2nC2 = 2n(n-1)/2
   = 23(3-1)/2
   = 23
   = 8
These 8 graphs are as shown below −

Eight Graphs
Connected Graph
A graph G is said to be connected if there exists a path between every pair of vertices. There should be at least one edge for every vertex in the graph. So that we can say that it is connected to some other vertex at the other side of the edge.

Example
In the following graph, each vertex has its own edge connected to other edge. Hence it is a connected graph.

\subsection{Connected Graph/Disconnected Graph}
A graph G is disconnected, if it does not contain at least two connected vertices.

Example 1
The following graph is an example of a Disconnected Graph, where there are two components, one with ‘a’, ‘b’, ‘c’, ‘d’ vertices and another with ‘e’, ’f’, ‘g’, ‘h’ vertices.

\subsection{Disconnected Graph}
The two components are independent and not connected to each other. Hence it is called disconnected graph.

Example 2
Disconnected Graph 1
In this example, there are two independent components, a-b-f-e and c-d, which are not connected to each other. Hence this is a disconnected graph.

\subsection{Regular Graph}
A graph G is said to be regular, if all its vertices have the same degree. In a graph, if the degree of each vertex is ‘k’, then the graph is called a ‘k-regular graph’.

Example
In the following graphs, all the vertices have the same degree. So these graphs are called regular graphs.

\subsection{Regular Graph}
In both the graphs, all the vertices have degree 2. They are called 2-Regular Graphs.

Complete Graph
A simple graph with ‘n’ mutual vertices is called a complete graph and it is denoted by ‘Kn’. In the graph, a vertex should have edges with all other vertices, then it called a complete graph.

In other words, if a vertex is connected to all other vertices in a graph, then it is called a complete graph.

Example
In the following graphs, each vertex in the graph is connected with all the remaining vertices in the graph except by itself.

Complete Graph
In graph I,

a	b	c
a	Not Connected	Connected	Connected
b	Connected	Not Connected	Connected
c	Connected	Connected	Not Connected
In graph II,

p	q	r	s
p	Not Connected	Connected	Connected	Connected
q	Connected	Not Connected	Connected	Connected
r	Connected	Connected	Not Connected	Connected
s	Connected	Connected	Connected	Not Connected
Cycle Graph
A simple graph with ‘n’ vertices (n >= 3) and ‘n’ edges is called a cycle graph if all its edges form a cycle of length ‘n’.

If the degree of each vertex in the graph is two, then it is called a Cycle Graph.

Notation − Cn

Example
Take a look at the following graphs −

Graph I has 3 vertices with 3 edges which is forming a cycle ‘ab-bc-ca’.

Graph II has 4 vertices with 4 edges which is forming a cycle ‘pq-qs-sr-rp’.

Graph III has 5 vertices with 5 edges which is forming a cycle ‘ik-km-ml-lj-ji’.

Cycle Graph
Hence all the given graphs are cycle graphs.

\subsection{Wheel Graph}
A wheel graph is obtained from a cycle graph Cn-1 by adding a new vertex. That new vertex is called a Hub which is connected to all the vertices of Cn.

Notation − Wn

No. of edges in Wn = No. of edges from hub to all other vertices +
                     No. of edges from all other nodes in cycle graph without a hub.
                     = (n–1) + (n–1)
                     = 2(n–1)
Example
Take a look at the following graphs. They are all wheel graphs.

\subsection{Wheel Graph}
In graph I, it is obtained from C3 by adding an vertex at the middle named as ‘d’. It is denoted as W4.

Number of edges in W4 = 2(n-1) = 2(3) = 6
In graph II, it is obtained from C4 by adding a vertex at the middle named as ‘t’. It is denoted as W5.

Number of edges in W5 = 2(n-1) = 2(4) = 8
In graph III, it is obtained from C6 by adding a vertex at the middle named as ‘o’. It is denoted as W7.

Number of edges in W4 = 2(n-1) = 2(6) = 12
Cyclic Graph
A graph with at least one cycle is called a cyclic graph.

Example
Cyclic Graph
In the above example graph, we have two cycles a-b-c-d-a and c-f-g-e-c. Hence it is called a cyclic graph.

\subsection{Acyclic Graph}
A graph with no cycles is called an acyclic graph.

Example
Acyclic Graph
In the above example graph, we do not have any cycles. Hence it is a non-cyclic graph.

\subsection{Bipartite Graph}
A simple graph G = (V, E) with vertex partition V = {V1, V2} is called a bipartite graph if every edge of E joins a vertex in V1 to a vertex in V2.

In general, a Bipertite graph has two sets of vertices, let us say, V1 and V2, and if an edge is drawn, it should connect any vertex in set V1 to any vertex in set V2.

Example
Bipartite Graph
In this graph, you can observe two sets of vertices − V1 and V2. Here, two edges named ‘ae’ and ‘bd’ are connecting the vertices of two sets V1 and V2.

\subsection{Complete Bipartite Graph}
A bipartite graph ‘G’, G = (V, E) with partition V = {V1, V2} is said to be a complete bipartite graph if every vertex in V1 is connected to every vertex of V2.

In general, a complete bipartite graph connects each vertex from set V1 to each vertex from set V2.

\subsection{Example}
The following graph is a complete bipartite graph because it has edges connecting each vertex from set V1 to each vertex from set V2.

Complete Bipartite Graph
If |V1| = m and |V2| = n, then the complete bipartite graph is denoted by Km, n.

Km,n has (m+n) vertices and (mn) edges.

Km,n is a regular graph if m=n.

In general, a complete bipartite graph is not a complete graph.

Km,n is a complete graph if m=n=1.

The maximum number of edges in a bipartite graph with n vertices is

[
n2
4
]
If n=10, k5, 5= ⌊
n2
4
⌋ = ⌊
102
4
⌋ = 25

Similarly K6, 4=24

K7, 3=21

K8, 2=16

K9, 1=9

If n=9, k5, 4 = ⌊
n2
4
⌋ = ⌊
92
4
⌋ = 20

Similarly K6, 3=18

K7, 2=14

K8, 1=8

‘G’ is a bipartite graph if ‘G’ has no cycles of odd length. A special case of bipartite graph is a star graph.

\subsection{Star Graph}
A complete bipartite graph of the form K1, n-1 is a star graph with n-vertices. A star graph is a complete bipartite graph if a single vertex belongs to one set and all the remaining vertices belong to the other set.

Example
Star Graph
In the above graphs, out of ‘n’ vertices, all the ‘n–1’ vertices are connected to a single vertex. Hence it is in the form of K1, n-1 which are star graphs.

\subsection{Complement of a Graph}
Let 'G−' be a simple graph with some vertices as that of ‘G’ and an edge {U, V} is present in 'G−', if the edge is not present in G. It means, two vertices are adjacent in 'G−' if the two vertices are not adjacent in G.

If the edges that exist in graph I are absent in another graph II, and if both graph I and graph II are combined together to form a complete graph, then graph I and graph II are called complements of each other.

Example
In the following example, graph-I has two edges ‘cd’ and ‘bd’. Its complement graph-II has four edges.

\subsection{Complement of Graph}
Note that the edges in graph-I are not present in graph-II and vice versa. Hence, the combination of both the graphs gives a complete graph of ‘n’ vertices.

Note − A combination of two complementary graphs gives a complete graph.

If ‘G’ is any simple graph, then

|E(G)| + |E('G-')| = |E(Kn)|, where n = number of vertices in the graph.

Example
Let ‘G’ be a simple graph with nine vertices and twelve edges, find the number of edges in 'G-'.

You have, |E(G)| + |E('G-')| = |E(Kn)|

12 + |E('G-')| = 

9(9-1) / 2 = 9C2
12 + |E('G-')| = 36

|E('G-')| = 24

‘G’ is a simple graph with 40 edges and its complement 'G−' has 38 edges. Find the number of vertices in the graph G or 'G−'.

Let the number of vertices in the graph be ‘n’.

We have, |E(G)| + |E('G-')| = |E(Kn)|

40 + 38 = 
n(n-1)
2

156 = n(n-1)

13(12) = n(n-1)

n = 13

\newpage
% https://www.tutorialspoint.com/graph_theory/graph_theory_trees.htm

Trees are graphs that do not contain even a single cycle. They represent hierarchical structure in a graphical form. Trees belong to the simplest class of graphs. Despite their simplicity, they have a rich structure.

Trees provide a range of useful applications as simple as a family tree to as complex as trees in data structures of computer science.
