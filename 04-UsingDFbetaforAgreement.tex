\documentclass[Main.tex]{subfiles}
\begin{document}
%-------------------------------------------------------------------------------------------------------------------------------------%
	\subsection{Influential Observations : DFBeta and DFBetas}
	\subsection{Measures of Influence} 
	% DFBETA
	% DFFITS
	% PRESS
	The impact of an observation on a regression fitting can be determined by the difference between the estimated regression coefficient of a model with all observations and the estimated coefficient when the particular observation is deleted. DFBETA and DFFITS are well known measures of influenc. The measure DFBETA is the studentized value of this difference.	DFFITS is a statistical measured designed to a show how influential an observation is in a statistical model. DFFITS is closely related to the studentized residual.
	
	\begin{eqnarray}
	DFBETA_{a} &=& \hat{\beta} - \hat{\beta}_{(a)} \\
	&=& B(Y-Y_{\bar{a}}
	\\ DFFITS = {\widehat{y_i} -
		\widehat{y_{i(k)}} \over s_{(k)} \sqrt{h_{ii}}} 
	\end{eqnarray}
	
	The prediction residual sum of squares (PRESS) is an value associated with this calculation. When fitting linear models, PRESS can be used as a criterion for model selection, with smaller values indicating better model fits.
	\begin{displaymath}
	PRESS = \sum(y-y^{(k)})^2
	\end{displaymath}
	%	
	%	\begin{itemize}
	%		\item $e_{-Q} = y_{Q} - x_{Q}\hat{\beta}^{-Q}$
	%		\item $PRESS_{(U)} = y_{i} - x\hat{\beta}_{(U)}$
	%	\end{itemize}
	%	
	
	\newpage
	% http://stats.stackexchange.com/questions/22161/how-to-read-cooks-distance-plots
	% Cook's distance refers to how far, on average, predicted y-values will move if the observation in question is dropped from the data set. 
	The dfbeta refers to how much a parameter estimate changes if the observation or case in question is dropped from the data set.  Cook's distance is presumably more important to you if you are doing predictive modeling, whereas dfbeta is more important in explanatory modeling.
	\subsection{DFBETA} %1.16.3
	DFBETAS (standardized difference of the beta) is a measure that standardizes the absolute difference in parameter estimates between a (mixed effects) regression model based on a full set of data, and a model from which a (potentially influential) subset of data is removed.
	%% -\subsection{DFBETA} %1.16.3
	\texttt{dfbeta()}
	
	
	The DFBETAS statistics are the scaled measures of the change in each parameter estimate and are calculated by deleting the th observation:
	
	where  is the th element of .
	In general, large values of DFBETAS indicate observations that are influential in estimating a given parameter. Belsley, Kuh, and Welsch (1980) recommend 2 as a general cutoff value to indicate influential observations and  as a size-adjusted cutoff.
	
	\begin{eqnarray}
	DFBETA_{a} &=& \hat{\beta} - \hat{\beta}_{(a)} \\
	&=& B(Y-Y_{\bar{a}}
	\end{eqnarray}
	In the case of method comparison studies, there are two covariates, and one can constru
	ct catterplots of the pairs of dfbeta values accordingly, both for LOO and LSO calculations. Furthermore 95\% confidence ellipse can be constructed around these scatterplots.
	Note that with k covariates, there will be $k+1$ dfbetas (the intercept,$\beta_0$, and 1 $\beta$ for each covariate). For example there would be 2 sets of of dfbeta, 510 valoues for each in the case of LOO, and 85 for LSO diagnostics.
	
	
	%-------------------------------------------------------------------------------------------------------------------------------------%
	
	\subsection{DFFITS} %1.16.1
	DFFITS is a statistical measured designed to a show how influential an observation is in a statistical model. 
	\begin{displaymath} DFFITS = {\widehat{y_i} -
		\widehat{y_{i(k)}} \over s_{(k)} \sqrt{h_{ii}}} \end{displaymath}
	It is closely related to the studentized residual. For the sake of brevity, we will concentrate on the Studentized Residuals.
	%===================================================================================== %
\end{document}