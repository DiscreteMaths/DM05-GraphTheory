Variance Ratios
The approach proposed by Roy deals with the question of agreement, and indeed interchangeability, as developed by Bland and Altman’s corpus of work.  In the view of Dunn, a question relevant to many practitioners is which of the two methods is more precise.
The relationship between precision and the within-item and between-item variability must be established. Roy establishes the equivalence of repeatability and within-item variability, and hence precision.  The method with the smaller within-item variability can be deemed to be the more precise.
A useful approach is to compute the confidence intervals for the ratio of within-item standard deviations (equivalent to the ratio of repeatability coefficients), which can be interpreted in the usual manner.  
In fact, the ratio of within-item standard deviations, with the attendant confidence interval,  can be determined using a single R command: intervals().
Pinheiro and Bates (pg 93-95) give a description of how confidence intervals for the variance components are computed. Furthermore a complete set of confidence intervals can be computed to complement the variance component estimates. 
What is required is the computation of the variance ratios of within-item and between-item standard deviations.  
A naïve approach would be to compute the variance ratios by relevant F distribution quantiles. However, the question arises as to the appropriate degrees of freedom.
Limits of agreement are easily computable using the LME framework. While we will not be considering this analysis, a demonstration will be provided in the example.
