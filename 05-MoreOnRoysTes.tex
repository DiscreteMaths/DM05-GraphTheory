\documentclass[Main.tex]{subfiles}
\begin{document}
%========================================================================== %
\section{Extension of Roy's methodology}
Roy's methodology is constructed to compare two methods in the presence of replicate measurements. Necessarily it is worth examining whether this methodology can be adapted for different circumstances.

An implementation of Roy's methodology, whereby three or more methods are used, is not feasible due to computational restrictions. Specifically there is a failure to reach convergence before the iteration limit is reached. This may be due to the presence of additional variables, causing the problem of non-identifiability. In the case of two variables, it is required to estimate two variance terms and four correlation terms, six in all. For the case of three variabilities, three variance terms must be estimated as well as nine correlation terms, twelve in all. In general for $n$ methods has $2 \times T_{n}$ variance terms, where $T_n$ is the triangular number for $n$, i.e. the addition analogue of the factorial. Hence the computational complexity quite increases substantially for every increase in $n$.

Should an implementation be feasible, further difficulty arises when interpreting the results. The fundamental question is whether two methods have close agreement so as to be interchangeable. When three methods are present in the model, the null hypothesis is that all three methods have the same variability relevant to the respective tests. The outcome of the analysis will either be that all three are interchangeable or that all three are not interchangeable.

The tests would not be informative as to whether any two of those three were interchangeable, or equivalently if one method in particular disagreed with the other two. Indeed it is easier to perform three pair-wise comparisons separately and then to combine the results.


\subsection{Roy's methodology for single measurements}
Roy's methodology follows from the decomposition for the covariance matrix of the response vector $y_{i}$, as presented in \citet{hamlett}. The decomposition depends on the estimation of correlation terms, which would be absent in the single measurement case. Indeed there can be no within-subject variability if there are no repeated terms for it to describe. There would only be the covariance matrix of the measurements by both methods, which doesn't require the use of LME models. To conlude, simple existing methodologies would be the correct approach where there only one measurements by each method.
Roy's methodology is not suitable for the case of single measurements because it follows from the decomposition for the covariance matrix of the response vector $y_{i}$, as presented in \citet{hamlett}. The decomposition depends on the estimation of correlation terms, which would be absent in the single measurement case. Indeed there can be no within-subject variability if there are no repeated terms for it to describe. There would only be the covariance matrix of the measurements by both methods, which doesn't require the use of LME models. To conclude, simpler existing methodologies, such as Deming regression, would be the correct approach where there only one measurements by each method.
\end{document}