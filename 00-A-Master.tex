\documentclass[]{report}

\voffset=-1.5cm
\oddsidemargin=0.0cm
\textwidth = 480pt

\usepackage{framed}
\usepackage{subfiles}
\usepackage{graphics}
\usepackage{newlfont}
\usepackage{eurosym}
\usepackage{amsmath,amsthm,amsfonts}
\usepackage{amsmath}
\usepackage{color}
\usepackage{enumerate}
\usepackage{amssymb}
\usepackage{multicol}
\usepackage[dvipsnames]{xcolor}
\usepackage{graphicx}
\begin{document}
%-------------------------------------------------------- %
\begin{verbatim}
5 Introduction to Graph Theory 
5.1 What is a graph?
5.1.1 Some definitions  
5.1.2 Degree of a vertex 
5.1.3 Some special graphs  

5.2 Paths, cycles and connectivity  
5.2.1 Paths 
5.2.2 Cycles  
5.2.3 Connectivity  

5.3 Isomorphism of graphs  
5.3.1 Showing that two graphs are isomorphic 
5.3.2 Showing that two graphs are not isomorphic
 
5.4 Adjacency matrices and adjacency lists  
5.4.1 Adjacency matrix of a graph 
5.4.2 Adjacency lists 

\end{verbatim}
\newpage
\section*{Graph Theory}
\subsection*{What is Graph Theory?}

Graph theory is the study of points and lines. In particular, it involves the ways in which sets of points, called \textit{\textbf{vertices}}, can be connected by lines or arcs, called \textit{\textbf{edges}}.

Graphs are classified according to their complexity, the number of edges allowed between any two vertices, and whether or not directions (for example, up or down) are assigned to edges. 

Graph theory is the study of points and lines. In particular, it involves the ways in which sets of points, called vertices, can be connected by lines or arcs, called edges. Graphs in this context differ from the more familiar coordinate plots that portray mathematical relations and functions.

Graphs are classified according to their complexity, the number of edges allowed between any two vertices, and whether or not directions (for example, up or down) are assigned to edges. Various sets of rules result in specific properties that can be stated as theorems.

Graph theory has proven useful in the design of integrated circuits ( IC s) for computers and other electronic devices. These components, more often called chip s, contain complex, layered microcircuits that can be represented as sets of points interconnected by lines or arcs. 

Using graph theory, engineers develop chips with maximum component density and minimum total interconnecting conductor length. This is important for optimizing processing speed and electrical efficiency.
%==================================================%

\begin{itemize}
\item Graph theory is the study of points and lines. In particular, it involves the ways in which sets of points, called vertices, can be connected by lines or arcs, called edges. Graphs in this context differ from the more familiar coordinate plots that portray mathematical relations and functions.

\item Graphs are classified according to their complexity, the number of edges allowed between any two vertices, and whether or not directions (for example, up or down) are assigned to edges. Various sets of rules result in specific properties that can be stated as theorems.

\item Graph theory has proven useful in the design of integrated circuits ( IC s) for computers and other electronic devices. These components, more often called chip s, contain complex, layered microcircuits that can be represented as sets of points interconnected by lines or arcs. 

\item Using graph theory, engineers develop chips with maximum component density and minimum total interconnecting conductor length. This is important for optimizing processing speed and electrical efficiency.
\end{itemize}

\section*{What is Graph Theory}
Graph theory is the study of graphs, which are mathematical structures used to model pairwise relations between objects. A "graph" in this context is made up of "vertices" or "nodes" and lines called edges that connect them. A graph may be undirected, meaning that there is no distinction between the two vertices associated with each edge, or its edges may be directed from one vertex to another

%-----------------------------------------------------%
\section*{Session 05:Graphs}
\begin{itemize}
\item[5A.1] What is a Graph?
\item[5A.2] Paths Cycles and Connectivity
\item[5A.3] Isomorphisms of a graph
\item[5A.4] Adjacency Matrices and Adjacency Lists
\end{itemize}


Graph theory is the study of points and lines. In particular, it involves the ways in which sets of points, called \textit{\textbf{vertices}}, can be connected by lines or arcs, called \textit{\textbf{edges}}.

Graphs are classified according to their complexity, the number of edges allowed between any two vertices, and whether or not directions (for example, up or down) are assigned to edges. 

%Various sets of rules result in specific properties that can be stated as theorems.


%-------------------------------------------------------- %

%-------------------------------------------------------- %
\newpage
Graph theory is the study of points and lines. In particular, it involves the ways in which sets of points, called vertices, can be connected by lines or arcs, called edges. Graphs in this context differ from the more familiar coordinate plots that portray mathematical relations and functions.

Graphs are classified according to their complexity, the number of edges allowed between any two vertices, and whether or not directions (for example, up or down) are assigned to edges. Various sets of rules result in specific properties that can be stated as theorems.

Graph theory has proven useful in the design of integrated circuits ( IC s) for computers and other electronic devices. These components, more often called chip s, contain complex, layered microcircuits that can be represented as sets of points interconnected by lines or arcs. 

Using graph theory, engineers develop chips with maximum component density and minimum total interconnecting conductor length. This is important for optimizing processing speed and electrical efficiency.
%------------------------------------------------------- %
\section*{Adjacency}
\begin{itemize}

\item[(a)] An \textit{\textbf{adjacency list}} representation of a graph is a collection of unordered lists, one for each vertex in the graph. Each list describes the set of neighbors of its vertex.

\item[(b)] An \textit{\textbf{adjacency matrix}} is a means of representing which vertices (or nodes) of a graph are adjacent to which other vertices. Another matrix representation for a graph is the incidence matrix.

\item[(c)]

\end{itemize}


\subsection*{Isomorphism}
\begin{itemize}
\item They have a different number of connected components
\item They have a different number of vertices
\item They have different degrees sequences
\item They have a different number of paths of any given length
\item They have a different number of cycles of any length.
\end{itemize}



%---------------------------- %

\begin{itemize}
\item Spanning Subgraphs of G.

\item a vertex is said to be an \textbf{emph{ isolated vertex}} if it has a degree of zero.
\item a vertex is said to be an \textbf{emph{ end-vertex}} if it has a degree of one.
\item a vertex is said to be an \textbf{emph{ even vertex}} if it has a degree of an even number.
\item a vertex is said to be an \textbf{emph{ odd vertex}} if it has a degree of an odd number.


\item A graph is said to be \textbf{emph{k-regular}} if the degree of each vertex is $k$. 
\item Every Graph has an even number of odd vertices.
\item A cubic graph is a graph where every vertex has degree three.
\end{itemize}


\end{document}
