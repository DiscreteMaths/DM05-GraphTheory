Influence Analysis for Repeated Measures Data

% - Influence Analysis for Repeated Measures Data

October 6, 2013
By Wingfeet










 (This article was first published on Wiekvoet, and kindly contributed to R-bloggers)

 I am trying exercise 59.8 (page 5057) of the SAS/STAT Users Guide 12.3 in R. The interesting thing is that influence is investigated on subject level rather than individual level. The diagnostics in nlme does not do leave-subject-out, at least, not that I know of. MCMCglm hardly has any diagnostics. This does not mean no validation is possible, this is R, programming is not optional, but rather expected. Hence with a little bit of work it is possible to estimate PRESS, Cook's D and effects on fixed effects. From this it follows extensive validation is possible, provided we can extract the underlying variables from the model fit object.

Data
Data is same as exercise 59.2 (exercise in R).
r1 <- read.table(textConnection('
1 F 21.0 20.0 21.5 23.0
2 F 21.0 21.5 24.0 25.5
3 F 20.5 24.0 24.5 26.0
4 F 23.5 24.5 25.0 26.5
5 F 21.5 23.0 22.5 23.5
6 F 20.0 21.0 21.0 22.5
7 F 21.5 22.5 23.0 25.0
8 F 23.0 23.0 23.5 24.0
9 F 20.0 21.0 22.0 21.5
10 F 16.5 19.0 19.0 19.5
11 F 24.5 25.0 28.0 28.0
12 M 26.0 25.0 29.0 31.0
13 M 21.5 22.5 23.0 26.5
14 M 23.0 22.5 24.0 27.5
15 M 25.5 27.5 26.5 27.0
16 M 20.0 23.5 22.5 26.0
17 M 24.5 25.5 27.0 28.5
18 M 22.0 22.0 24.5 26.5
19 M 24.0 21.5 24.5 25.5
20 M 23.0 20.5 31.0 26.0
21 M 27.5 28.0 31.0 31.5
22 M 23.0 23.0 23.5 25.0
23 M 21.5 23.5 24.0 28.0
24 M 17.0 24.5 26.0 29.5
25 M 22.5 25.5 25.5 26.0
26 M 23.0 24.5 26.0 30.0
27 M 22.0 21.5 23.5 25.0
'),col.names=c('Person','Gender','Age8','Age10','Age12','Age14'),
colClasses=c('factor','factor',rep('numeric',4)))
rm <- reshape(r1,direction='long',
    varying=list(c('Age8','Age10','Age12','Age14')),
    timevar='Age',idvar=c('Person','Gender'),
    v.names='y',
    times=c(8,10,12,14))
rm$Gender <- relevel(rm$Gender,ref='M')
rm$fage=factor(rm$Age)
rm$Person <- factor(rm$Person,levels=format(1:27,trim=TRUE))
rm <- rm[order(rm$Person,rm$Age),]

nlme

Analysis, standard plot are not too difficult.


lSymm <- lme(y ~  Age * Gender,

    data=rm, random= list(Person =pdSymm(~ fage-1)),method='ML')




plot(lSymm, resid(., type = "p") ~ Age | Person)






Subject level influence

I have chosen to display three items, PRESS, Cook's D and effect on fixed parameters. PRESS is reasonable straightforward, the numbers more or less match. Cook's D on the other hand, is not. I took the formula of the SAS/STAT guide, which calculated it, my wording, as Mahalanobis distance of leave-subject-out fixed parameters. However, the numbers don't match. Part of that may be that I do recalculate random parameters too. Compare my figure with Output 59.8.3 top left, this seems quite similar. 


coefFulllme <- as.numeric(coef(lSymm)[1,1:4])

VCMlme <- vcov(lSymm)

lSymmLSO <- sapply(levels(rm$Person), function(x) {

      rloo <- rm[rm$Person !=x,]

      lSymm <- lme(y ~  Age * Gender,

          data=rloo, random= list(Person =pdSymm(~ fage-1)),

          method='ML')

      coef <- as.numeric(coef(lSymm)[1,1:4])

      genderF <- rm[rm$Person==x,'Gender'][1]=='F'

      pred <- coef[1]+

          c(8,10,12,14)*coef[2]+

          genderF*coef[3]+

          c(8,10,12,14)*coef[4]*genderF

      obs <- rm[rm$Person ==x,'y']

      CD=mahalanobis(coef,coefFulllme,VCMlme)/4

      c(press=sum((obs-pred)^2),cd=CD,coef)

    })

lSymmLSO <- t(lSymmLSO)

lSymmLSO[,c(1,2)]


       press          cd

1  10.323531 0.020169407

2   3.839932 0.043571109

3  10.899537 0.032741341

4  24.050823 0.045570019

5   1.690000 0.016239351

6  11.866688 0.016490548

7   1.191555 0.005405341

8   4.678438 0.027992197

9  13.488546 0.042166906

10 85.581754 0.155169369

11 68.465309 0.106829228

12 39.449016 0.022192784

13 12.920225 0.007427751

14  6.119853 0.001987303

15 26.126968 0.277545188

16 21.065821 0.018968667

17 10.050641 0.016763371

18  7.800734 0.007611981

19 15.196358 0.008490818

20 43.256052 0.395086129

21 96.107295 0.115497649

22 13.879678 0.033635096

23  4.961204 0.011979580

24 41.758892 0.905440147

25  4.643721 0.095851759

26  8.019698 0.026621476

27 19.920691 0.017942506


plot(y=lSymmLSO[,2],x=1:27,main="Cook's D",xlab='Subject',type='h')




Fixed-Effects Deletion Estimates

Having done all the pre-work, the fixed effects deletion statistics are just a plot away.They do look slightly different from PROC MIXED as the model is a bit different in the SAS/STAT Guide.


par(mfrow=c(2,2))

dummy <- sapply(1:4,function(x) {

  plot(y=lSymmLSO[,x+2],x=1:27,main=names(coef(lSymm))[x],ylab='',xlab='Subject')

  abline(h=coefFulllme[x])

})






MCMCglm

The model is easy enough to fit. The approach used in nlme is easy enough to convert. Only first 5 subject's data shown for brevity. 
library(MCMCglmm)
prior1 <- list(R=list(V=diag(4),nu=.01),
    G=list(G1=list(V=diag(1),nu=.01) ))

m1 <- MCMCglmm(y ~ Age* Gender , 
    random= ~ Person ,
    rcov=~ us(fage) :Person,
    data=rm,family='gaussian',
    nitt=500000,thin=20,burnin=50000,
    verbose=FALSE
 ,   prior=prior1
)

VCMM1 <- cov(m1$Sol)
coefFullM1 <- colMeans(m1$Sol)


m1LSO <- sapply(levels(rm$Person), function(x) {

      rloo <- rm[rm$Person !=x,]

      m1 <- MCMCglmm(y ~ Age* Gender , 

          random= ~ Person ,

          rcov=~ us(fage) :Person,

          data=rloo,family='gaussian',

     nitt=500000,thin=20,burnin=50000,

          verbose=FALSE

          ,   prior=prior1

      )

      coef <- colMeans(m1$Sol)

      genderF <- rm[rm$Person==x,'Gender'][1]=='F'

      pred <- coef[1]+

          c(8,10,12,14)*coef[2]+

          genderF*coef[3]+

          c(8,10,12,14)*coef[4]*genderF

      obs <- rm[rm$Person ==x,'y']

      CD=mahalanobis(coefFullM1,coef,VCMM1)/4

      c(press=sum((obs-pred)^2),cd=CD,coef)

    })

m1LSO <- t(m1LSO)

m1LSO[1:5,1:2]



      press         cd

1 10.133019 0.01117458

2  3.835048 0.03347745

3 10.891120 0.02740768

4 24.167881 0.03618814

5  1.694508 0.01317537
For both PRESS and Cook's D it is close. For completeness the plot.Not exactly the same, but the trend and interpretations are clear enough.

plot(y=m1LSO[,2]*4,x=1:27,main="Cook's D",xlab='Subject',type='h')








Fixed-Effects Deletion Estimates

Following the template from nlme, the result is similar. 


par(mfrow=c(2,2))

dummy <- sapply(1:4,function(x) {

      plot(y=m1LSO[,x+2],x=1:27,main=names(coefFullM1)[x],ylab='',xlab='Subject')

      abline(h=coefFullM1[x])

    })

