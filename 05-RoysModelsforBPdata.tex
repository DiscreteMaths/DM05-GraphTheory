\newpage
\section{Implementation in R}
To implement an LME model in \texttt{R}, the \texttt{nlme} package is used. This package is loaded into the \texttt{R} environment using the library command, (i.e.\ \texttt{library(nlme)}). The \texttt{lme} command is used to fit LME models. The first two arguments to the \texttt{lme} function specify the fixed effect component of the model, and the data set to which the model is to be fitted. The first candidate model (`MCS1') fits an LME model on the data set `dat'. The variable `method' is assigned as the fixed effect, with the response variable `BP' (i.e.\ blood pressure).

The third argument contain the random effects component of the formulation, describing the random effects, and their grouping structure. The \texttt{nlme} package provides a set of positive-definite matrices , the \texttt{pdMat} class, that can be used to specify a structure for the between-subject variance-covariance matrix for the random effects. For Roy's methodology, we will use the \texttt{pdSymm} and \texttt{pdCompSymm} to specify a symmetric structure and a compound symmetry structure respectively. A full discussion of these structures can be found in \citet[pg. 158]{PB}.

Similarly a variety of structures for the with-subject variance-covariance matrix can be implemented using \texttt{nlme}. To implement a particular matrix structure, one must specify both a variance function and correlation structure accordingly. Variance functions are used to model the variance structure of the within-subject errors. \texttt{varIdent} is a variance function object used to allow different variances according to the levels of a classification factor in the data. A compound symmetry structure is implemented using the \texttt{corCompSymm} class, while the symmetric form is specified by \texttt{corSymm} class. Finally, the estimation methods is specified as ``ML" or ``REML".
\newpage
The first of Roy's candidate model can be implemented using the following code;\\
\hrule
\begin{verbatim}
MCS1 = lme(BP ~ method-1, data = dat,
random =  list(subject=pdSymm(~ method-1)),
weights=varIdent(form=~1|method),
correlation = corSymm(form=~1 | subject/obs), method="ML")
\end{verbatim}
\hrule
\vspace{1cm}
For the blood pressure data used in \citet{roy}, all four candidate models are implemented by slight variations of this piece of code, specifying either \texttt{pdSymm} or \texttt{pdCompSymm} in the second line, and either \texttt{corSymm} or \texttt{corCompSymm} in the fourth line.
For example, the second candidate model `MCS2' is implemented with the same code as MCS1, except for the term \texttt{pdCompSymm} in the second line, rather than \texttt{pdSymm}.
\\
\hrule
\begin{verbatim}
MCS2 = lme(BP ~ method-1, data = dat,
random = list(subject=pdCompSymm(~ method-1)),
weights = varIdent(form=~1|method),
correlation = corSymm(form=~1 | subject/obs), method="ML")
\end{verbatim}
\hrule
\vspace{1cm}
Using this \texttt{R} implementation for other data sets requires that the data set is structured appropriately (i.e.\ each case of observation records the index, response, method and replicate). Once formatted properly, implementation is simply a case of re-writing the first line of code, and computing the four candidate models accordingly.
\newpage
To perform a likelihood ratio test for two candidate models, simply use the \texttt{anova} command with the names of the candidate models as arguments. The following piece of code implement the first of Roy's variability tests.
\\
\hrule
\begin{verbatim}
> anova(MCS1,MCS2)
Model df    AIC    BIC  logLik   Test L.Ratio p-value
MCS1     1  8 4077.5 4111.3 -2030.7
MCS2     2  7 4075.6 4105.3 -2030.8 1 vs 2 0.15291  0.6958
>
\end{verbatim}
\hrule
\vspace{1cm}
The fixed effects estimates are the same for all four candidate models. The inter-method bias can be easily determined by inspecting a summary of any model. The summary presents estimates for all of the important parameters, but not the complete variance-covariance matrices (although some simple \texttt{R} functions can be written to overcome this). The variance estimates for the random effects for MCS2 is presented below.
\\
\hrule
\begin{verbatim}
Random effects:
Formula: ~method - 1 | subject
Structure: Compound Symmetry
StdDev Corr
methodJ  30.765
methodS  30.765 0.829
Residual  6.115
\end{verbatim}
\hrule
\vspace{1cm}
Similarly, for computing the limits of agreement the standard deviation of the differences is not explicitly given. Again, A simple \texttt{R} function can be written to calculate the limits of agreement directly.

%------------------------------------------------------------------------------------%

\section{Basic Models Fits}
Further to \citet{PB}, several simple LME models are constructed
for the blood pressure data. This data set is the subject of a
method comparison study in \citet{BA99}.

\subsection{Implementing the Mixed Models Fits}
They are implemented using the following {\tt{R}} code, utilising the
`nlme' package. An analysis of variance is used to compare the model fits.

The {\tt{R}} script:
\begin{verbatim}
fit1 = lme( BP ~ method, data = dat, random = ~1 | subject )
fit2 = update(fit1, random = ~1 | subject/method )
fit3 = update(fit1, random = ~method - 1 | subject )
#analysis of variance
anova(fit1,fit2,fit3)
\end{verbatim}


\begin{enumerate}
	
	
	\item Simplest workable model, allows differences between methods
	and incorporates a random intercept for each subject. For subject
	1 we have
	\[
	\boldsymbol{X}_i =
	\left(%
	\begin{array}{cc}
	1 & 0 \\
	1 & 0 \\
	1 & 0 \\
	1 & 1 \\
	1 & 1 \\
	1 & 1 \\
	\end{array}%
	\right),\quad
	\boldsymbol{\beta} =
	\left(%
	\begin{array}{c}
	\beta_0 \\
	\beta_1 \\
	\end{array}%
	\right), \quad
	\boldsymbol{Z}_i =
	\left(%
	\begin{array}{c}
	1 \\
	1 \\
	1 \\
	1 \\
	1 \\
	1 \\
	\end{array}%
	\right), \quad \boldsymbol{b}_i = b
	\]
	where $\mathrm{E}(b)=0$ and $\mathrm{var}(b)=\psi.$
	
	\item
	\[
	\boldsymbol{Z}_i =
	\left(%
	\begin{array}{c c}
	1 & 0 \\
	1 & 0 \\
	1 & 0 \\
	0 & 1 \\
	0 & 1 \\
	0 & 1 \\
	\end{array}%
	\right)
	\quad \boldsymbol{b}_i =
	\left(%
	\begin{array}{c c}
	b_1 & 0  \\
	0 & b_2  \\
	\end{array}%
	\right)
	\]
	
	where $\mathrm{E}(b_i)=0$ and $\mathrm{var}(\boldsymbol{b})=
	\boldsymbol{\Psi}$.
	
	The variance of error terms is a $6 \times 6$ matrix.
	
\end{enumerate}






\newpage
\subsection{Model Fit 1}

This is a simple model with no interactions. There is a fixed effect for each method and a random effect for each subject.
\begin{equation*}
	y_{ijk} = \beta_{j}  + b_{i} + \epsilon_{ijk}, \qquad i=1,\dots,2, j=1,\dots,85, k=1,\dots,3
\end{equation*}

\begin{eqnarray*}
	b_{i} \sim \mathcal{N}(0,\sigma^2_{b}), \qquad \epsilon_{i} \sim \mathcal{N}(0,\sigma^2)
\end{eqnarray*}

\begin{verbatim}
Linear mixed-effects model fit by REML
Data: dat
Log-restricted-likelihood: -2155.853
Fixed: BP ~ method
(Intercept)     methodS
127.40784    15.61961

Random effects:
Formula: ~1 | subject
(Intercept) Residual
StdDev:    29.39085 12.44454

Number of Observations: 510
Number of Groups: 85
\end{verbatim}

The following output was obtained.
\begin{verbatim}
Linear mixed-effects model fit by REML
Data: dat
Log-restricted-likelihood: -2047.582
Fixed: BP ~ method
(Intercept)     methodS
127.40784    15.61961

Random effects:
Formula: ~method - 1 | subject
Structure: General positive-definite, Log-Cholesky parametrization
StdDev    Corr
methodJ  30.455093 methdJ
methodS  31.477237 0.835
Residual  7.763666

Number of Observations: 510
Number of Groups: 85

\end{verbatim}
\newpage
\subsection{Model Fit 1}

This is a simple model with no interactions. There is a fixed effect for each method and a random effect for each subject.
\begin{equation*}
y_{ijk} = \beta_{j}  + b_{i} + \epsilon_{ijk}, \qquad i=1,\dots,2, j=1,\dots,85, k=1,\dots,3
\end{equation*}

\begin{eqnarray*}
b_{i} \sim \mathcal{N}(0,\sigma^2_{b}), \qquad \epsilon_{i} \sim \mathcal{N}(0,\sigma^2)
\end{eqnarray*}

\begin{verbatim}
Linear mixed-effects model fit by REML
  Data: dat
  Log-restricted-likelihood: -2155.853
  Fixed: BP ~ method
(Intercept)     methodS
  127.40784    15.61961

Random effects:
 Formula: ~1 | subject
        (Intercept) Residual
StdDev:    29.39085 12.44454

Number of Observations: 510
Number of Groups: 85
\end{verbatim}

The following output was obtained.
\begin{verbatim}
Linear mixed-effects model fit by REML
  Data: dat
  Log-restricted-likelihood: -2047.582
  Fixed: BP ~ method
(Intercept)     methodS
  127.40784    15.61961

Random effects:
 Formula: ~method - 1 | subject
 Structure: General positive-definite, Log-Cholesky parametrization
         StdDev    Corr
methodJ  30.455093 methdJ
methodS  31.477237 0.835
Residual  7.763666

Number of Observations: 510
Number of Groups: 85

\end{verbatim}

\newpage
\subsection{Model Fit 2}

This is a simple model, this time with an interaction effect.
There is a fixed effect for each method. This model has random effects at two levels $b_{i}$ for the subject, and
another, $b_{ij}$, for the respective method within each subject.
\begin{equation*}
	y_{ijk} = \beta_{j}  + b_{i} + b_{ij} + \epsilon_{ijk}, \qquad i=1,\dots,2, j=1,\dots,85, k=1,\dots,3
\end{equation*}
\begin{eqnarray*}
	b_{i} \sim \mathcal{N}(0,\sigma^2_{1}), \qquad b_{ij} \sim \mathcal{N}(0,\sigma^2_{2}), \qquad \epsilon_{i} \sim \mathcal{N}(0,\sigma^2)
\end{eqnarray*}

In this model, the random interaction terms all have the same variance $\sigma^2_{2}$. These terms are assumed to be independent of each other, even
within the same subject.

\begin{verbatim}
Linear mixed-effects model fit by REML
Data: dat
Log-restricted-likelihood: -2047.714
Fixed: BP ~ method
(Intercept)     methodS
127.40784    15.61961

Random effects:
Formula: ~1 | subject
(Intercept)
StdDev:    28.28452

Formula: ~1 | method %in% subject
(Intercept) Residual
StdDev:    12.61562 7.763666

Number of Observations: 510
Number of Groups:
subject method %in% subject
85                 170
\end{verbatim}
\newpage
\subsection{Model Fit 3}

This model is a more general model, compared to 'model fit 2'. This model treats the random interactions for each subject as a vector and
allows the variance-covariance matrix for that vector to be estimated from the set of all positive-definite matrices.
$\boldsymbol{y_{i}}$ is the entire response vector for the $i$th subject.
$\boldsymbol{X_{i}}$ and $\boldsymbol{Z_{i}}$  are the fixed- and random-effects design matrices respectively.
\begin{equation*}
	\boldsymbol{y_{i}} = \boldsymbol{X_{i}\beta}  + \boldsymbol{Z_{i}b_{i}} + \boldsymbol{\epsilon_{i}}, \qquad i=1,\dots,85
\end{equation*}
\begin{eqnarray*}
	\boldsymbol{Z_{i}} \sim \mathcal{N}(\boldsymbol{0,\Psi}),\qquad
	\boldsymbol{\epsilon_{i}} \sim \mathcal{N}(\boldsymbol{0,\sigma^2\Lambda})
\end{eqnarray*}

For the first subject the response vector, $\boldsymbol{y_{1}}$, is:
\begin{table}[ht]
	\begin{center}
		\begin{tabular}{rrllr}
			\hline
			observation & BP & subject & method & replicate \\
			\hline
			1 & 100.00 & 1 & J &   1 \\
			86 & 106.00 & 1 & J &   2 \\
			171 & 107.00 & 1 & J &   3 \\
			511 & 122.00 & 1 & S &   1 \\
			596 & 128.00 & 1 & S &   2 \\
			681 & 124.00 & 1 & S &   3 \\
			\hline
		\end{tabular}
	\end{center}
\end{table}
\newpage
The fixed effects design matrix $\boldsymbol{X_{i}}$ is given by:
\begin{table}[ht]
	\begin{center}
		\begin{tabular}{r|r}
			\hline
			(Intercept) & method S \\
			\hline
			1 & 0 \\
			1 & 0 \\
			1 & 0 \\
			1 & 1 \\
			1 & 1 \\
			1 & 1 \\
			\hline
		\end{tabular}
	\end{center}
\end{table}

\newpage




The random effects design matrix $\boldsymbol{Z_{i}}$ is given by:
\begin{table}[ht]
	\begin{center}
		\begin{tabular}{r|r}
			\hline
			method J & method S \\
			\hline
			1 & 0 \\
			1 & 0 \\
			1 & 0 \\
			0 & 1 \\
			0 & 1 \\
			0 & 1 \\
			\hline
		\end{tabular}
	\end{center}
\end{table}
\newpage
